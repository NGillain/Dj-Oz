\documentclass[a4paper,12pt]{article}

\usepackage[francais]{babel}
\usepackage[T1]{fontenc}
\usepackage[utf8]{inputenc}
\usepackage{lmodern}
\usepackage{graphicx}
\usepackage{ulem}
\usepackage{enumerate}

\title{Rapport Projet FSA1402 : Dj-Oz}
\pagestyle{plain}
\author{Gillain Nathan : NOMA 78791200 \and Hoo Sing Leung : NOMA }

\begin{document}
\maketitle
\tableofcontents

% Ne pas oublier de préciser le fait que l'on n'utilise seulement le modèle déclaratif

\section{Structure du programme et décisions de conceptions}

Dans cette section, nous détaillerons l'algorithme qui nous a amener à coder nos fonctions telles que nous les avons codées et
nous nous arrêterons sur les points où il a fallu choisir un algorithme plutôt qu'un autre. Point important que nous tenons 
souligner : nous n'avons utilisé que des structures du paradigme de programmation déclaratif. Nous n'avons pas juger utile
d'utiliser d'autres structures car la puissance de ce paradigme est assez impresionnante (malgré quelques petites limitations).
Nous avons donc utiliser beaucoup de pattern matching et de récursion sur des fonctions que nous avons ensuite cachées grâce
à la programmation d'ordre supérieur dans d'autres fonctions pou rendre le tout propre. Bref, voici la structure de notre programme.

\subsection{La fonction Interprete}

Après avoir étudier l'énoncé du projet, nous nous sommes lancés dans la conception de la fonction Interprète car nous avons vu
que la fonction Mix nécessitait cette fonction. Nous avons donc jugé qu'il était préférable de commencer par cette fonction.
Nous avons tout d'abord traiter l'argument que prennait Interprete : c'est une partition. Dès lors, nous nous sommes posés la 
question comment transformer une liste d'atome en un résultat exploitable pour la fonction Mix. Nous avons alors décider 
de mettre la partition dans une liste d'échantillons. Un échantillon étant un enregistrement qui prend comme champ la hauteur
de la note, sa durée et son instrument. Ensuite, nous avons utilisé du pattern matching pour traiter chaque élément de la 
partition (note, transformation, suite de partitions). 
Dans le premier cas, nous avons transformer la note en une note étendue avec la méthode ToNote. Une fois effectué, on appelle 
la fonction NoteToEchantillon (fonction qui prend en argument une note étendue et qui renvoit un échantillon).
Dans le cas où nous avons une transformation, on utilise du pattern matching pour voir de quelle transformation il s'agit. 
Nous appelons ensuite la fonction qui lui correspond. Au niveau des différentes transformations, nous avons un algorithme 
semblable pour chaque fonction. En effet, chaque fonction prend deux arguments. Le premier correspond à une liste 
d'échantillons tandis que le deuxième correspond au changement à devoir effectuer. Mis à part pour la fonction Duree où nous 
appelons une autre fonction DureePart qui calcule la durée totale de la partition avec l'aide d'un accumulateur.
Enfin, pour traiter le cas de suite de partitions, nous utilisons la méthode de récursion terminale.

\subsection{Le fonction Mix}

Après avoir transformé une liste d'atomes représentant les notes d'une partition standard en liste de record que l'on peut
mieux exploiter, la deuxième fonction, à savoir {Mix Interprete Music}, va utiliser cette liste de record et l'échantillonner
à $44100$ valeurs sur un sinus afin d'obtenir un vecteur audio que l'ordinateur peut utiliser grâce au code annexe.

Pour commencer, la fonction Mix fait du pattern matching afin de déterminer quel type d'argument Mix doit traiter.
Si c'est une voix, Il utilise VoiceToVector qui transforme une liste d'echantillon en vecteur audio. Cette fonction
utilise une autre fonction appelée ToVector qui transforme un seul echantillon en vecteur audio. En utilisant la récursion,
nous pouvons écrire VoiceToVector de manière simple et efficace. Si l'argument de Mix est une partition, nous devons d'abord
l'interpreter avec la fonction Interprete puis lui appliquer VoiceToVector étant donné que la partition interpretée est une voix.
Si c'est un fichier au format .WAV, une fonction est fournie par le professeur pour lire le fichier. Elle s'appelle
Projet.readFile. Ensuite, si l'argument est un record suivant la pattern "merge(VecteurAudio)", alors on applique la fonction
MergeToVector. Cette fonction utilise deux sous-fonction, ListVector et AddVector, et une fonction interne Somme. Somme est simple :
il utilise AddVector par récursion afin d'additionner des vecteurs même s'il sont de tailles différentes
(en rajoutant du silence si nécessaire). ListVector, quand à elle, crée à partir d'une liste d'atomes
avec la pattern "coefficient#musique" une liste de vecteurs audio pondéré par les coefficient de la liste entrée en argument.
Donc, MergeToVector utilise d'abord ListVector pour avoir une liste de vecteurs audio à superposer et Somme,
qui utilise AddVector, retourne un vecteur audio contenant les musiques pondérées.

Enfin, Si l'argument ne correspond à aucun des pattern ci-dessus, c'est que celui-ci est un filtre. Nous avons décidé de créer
un dispatcher(appelé Dispatcher *très original*) qui, grâce au pattern matching va déterminer de quel filtre il s'agit et va donc
traiter le morceau avec la fonction appropriée. Ici, nous avons six filtres dont certains on plusieurs signatures différentes :
renverser, repetition, clip, echo, fondu et couper. Renverser retourne le vecteur audio à l'envers, ce qui permet de créer des messages
subliminaux géniaux. Pour cela, nous utilisons un accumulateur et ajoutons le premier élément obtenu grâce au pattern matching
dans l'accumulateur en tête de la liste. Pour Repetition, nous utilisons la récursion pour rejouer le même vecteur audio plusieurs fois
ou jusqu'à un certain temps. Le filtre clip, quant à lui, plafonne les valeurs minimales et maximales du vecteur audio (par récursion),
c'est-à-dire que si la valeur est en supérieure/inférieur aux limites, alors on prend la valeur maximale/minimale et on la met
dans le vecteur audio sinon on laisse la valeur actuelle.  Ensuite, Echo produit un echo dans la musique avec un delai ou delai et
une decadence voire un delai, une decadence et une repretition. Pour cette fonction, nous avons décider dans un premier temps de retourner
une liste d'atomes au pattern "coefficient#musique" que nous utiliserons en argument dans merge par la suite pour obtenir un seul vecteur audio.
En ce qui concerne fondu, cette fonction adoucit la musique au début et à la fin. Pour cela, il suffit de multiplier les valeurs du début et
de la fin par un facteur qui réduit correctement l'intensité du vecteur audio pour faire un fondu. Fondu_Enchaine applique un fondu sur la fin
d'un morceau et sur le début d'un autre morceau et concatene les deux vecteurs sur leur fondu. Le principe est le même que pour le fondu :
on multiplie les valeurs par un coefficient afin de diminuer l'intensité et obtenir un vecteur adoucit. Enfin, le filtre Couper prend une partie
du morceau en fonction de $S1$ et $S2$. Si $S1$ et $S2$ sont hors du morceau, alors on rajoute du silence. Voila qui clôt l'algorithme de Mix.

\section{Difficultés et limitations du programme}

Dans cette section, nous allons détailler les difficultés rencontrées lors de la conception du programme

Nous avons éprouvé des difficultés pour comprendre les objectifs de Interprete et Mix au début. La synthaxe était fort lourde ce qui fait que
l'on se perdait souvent par soucis de compréhension des notations. De plus, l'implémentation des filtres a été un peu ardue. En effet,
nous avons testé notre code fonction par fonction et comparé le résultat obtenu avec le résultat attendu. Souvent, notre algorithme de base était correct
mais l'implémentation l'était un peu moins, ce qui fait que nous n'arrivions pas toujours au bon résultat. Dans la plupart des cas où les résulatts différaient,
nous comprenions ce qui ne fonctionnait pas, par exemple, un indice "OutOfBound" ou erreur "Float attendu". Ce qui fait que nous corrigions notre implémentation
au fur et à mesure pour chaque cas traité dans la fonction et que, finalement, nous arrivions au résultat final présenté plus haut. Le reste du programme
n'a pas posé de problème majeurs : Interprete était somme toute assez simple (malgré qu'il fallait comprendre la synthaxe) et Mix était similaire dans le sens
où il fallait, pour les deux fonctions, appliqué du pattern matching pour connaître le type d'arguments introduits dans la fonction.

Nous avons essayé d'optimiser le coût en mémoire et en opérations en rendant nos fonction récursives terminales. Cela n'a pas toujours été possible,
notamment pour une fonction, où nos devions Flatten une liste décomposée esleon le pattern H|T et chaque H et T devaient être calculé par récursion,
ce qui augmente la taille de la pile sémantique. Cependant, l'opération Flatten est obligatoire afin d'obtenir une liste sans liste a l'intérieur.

\section{Complexité des fonctions du programme}

\section{Extension(s) apportée(s)}

\end{document}
