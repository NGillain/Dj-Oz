\documentclass[a4paper,12pt]{article}

\usepackage[francais]{babel}
\usepackage[T1]{fontenc}
\usepackage[utf8]{inputenc}
\usepackage{lmodern}
\usepackage{graphicx}
\usepackage{ulem}
\usepackage{enumerate}

\title{Rapport Projet FSA1402 : Dj-Oz}
\pagestyle{plain}
\author{Gillain Nathan : NOMA 78791200 \and Hoo Sing Leung : NOMA }

\begin{document}
\maketitle
\tableofcontents

% Ne pas oublier de pr�ciser le fait que l'on n'utilise seulement le mod�le d�claratif

\section{Structure du programme et d�cisions de conceptions}

Dans cette section, nous d�taillerons l'algorithme qui nous a amener � coder nos fonctions telles que nous les avons cod�es et nous nous arr�terons sur les points o� il a fallu choisir un algorithme plut�t qu'un autre.

\subsection{La fonction Interprete}

\subsection{Le fonction Mix}

\section{Difficult�s et limitations du programme}

\section{Complexit� des fonctions du programme}

\section{Extension(s) apport�e(s)}

\end{document}